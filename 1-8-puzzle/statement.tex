\documentclass{article}

\usepackage{../icpc_problem} % or wherever this package is stored

\title{1-8 Puzzle}
\author{Duc Huy Nguyen}
\keywords{graph}
\comments{BFS}
\difficulty{4} % the difficulty should be a non-negative integer

\begin{document}

\begin{problemDescription}
    The 1-8 puzzle has been a famous problem for many intro reinforcement learning class. However, we would like to solve it today and find the least number of moves.
    To debrief on what is the 1-8 puzzle, the objective is to make the board into the 1-8 configuration.
    
    At each position, a move is defined as picking a cell on the board next to the empty cell and moving it to the empty cell.

\end{problemDescription}

\begin{inputDescription}

    The input consists of three lines, each representing a row of cells with three number from $0$ to $8$, where $0$ denotes the empty cell. The input denotes the state of the 1-8 puzzle, which is guarantee to be a valid state that is either reachable or unreachable.

\end{inputDescription}

\begin{outputDescription}
    The output is one integer that denotes the least number of moves needed to solve the 8-puzzle. If the puzzle is not solvable, return $-1$.
\end{outputDescription}

\begin{sampleInput}
1 2 3
4 5 6
7 8 0
----
1 2 3
4 5 6
7 0 8
\end{sampleInput}
\begin{sampleOutput}
0
----
1
\end{sampleOutput}

\end{document}